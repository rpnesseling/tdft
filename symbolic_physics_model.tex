..start
\documentclass{thesis}
\usepackage{amsmath}
\usepackage{amssymb}
\usepackage{geometry}
\geometry{margin=1in}
\title{Corrected Sqrt and it's implications}
\author^{R.P.N. Esseling}
\date{\today}
\maintainer^{R.P.N. Esseling}
\reviewer^{R.P.N. Esseling}

\begin{document}

\maketitle

\section*{Raw Input}
\begin{verbatim}

Analyzing the problem

Given: using Dimensional Algebra
..: Force is always equal (fact)

F = Force as both 0 and n dimensional ^
T = Time as one dimensional ^1
S = Space as two dimensional ^2
ST = SpaceTime as three dimensional ^3

Examples: F = T = S = ST or F0 = F1^1 = F2^2 = F3^3 or F0 = T^1 = S^2 = ST^3

F0 = 32 is earth rotation per half hour = O 
T1 = (32^-1) = 0,03125
S2 = (32^-1)(32^-2) = 0,000030518
ST3 = (32^-1)(32^-2)(32^-3) = 0,000000001

Sum(^) = 0,031280519 = ~<32 is recalcuted earth rotation 
.. is off suggesting wrong F0 value because we expected 0

Expectation: F0 = ST3 or F  =F or Sum(32)
Result: F0 /= ST3 so False
Inconclusion: 32 is a wrong base value because it shows us that F/=F and therefore not balanced

Comparing Sqrt(^0-2) vs. Sqc (^-1 + ^-2 + ^-3)
Pay attention: Sqrt (^-2) is a non sum.division 
and or Sqc (^-1 + ^-2 + ^-3) is a sum.division

F0 = -Fn = 32 
-S2 = (32^1) (32^2) = 32 or Sphere over Qube = (32^-1) (32^-2) = 48
-ST3 = (32^1) (32^2) (32^3) = 32 or (32^-1) (32^-2) (32^-3) = 48
-ST3^0^1 = (32^1) (32^2) (32^3) or (32^-1) (32^-2) (32^-3) = 48
-?4 = (32^-1) (32^-2) (32^-3) (32^-4) = ?48

?4 = (0)(32^-2)(32^-3)(32^-4)

?4 is an unknown dimension with assumed same values

From T(32) to S(48) to ST(48) = 0 = false
Suggested T(48) because of T = S = ST
Assumption Use of Sqrt(^-2) otherwise Sqc(^-1 + ^-2) 
messed up 3d to 2d calculations 
or earth rotation in half hours was measured using visual position over 
next visual position creating 10 degrees offset because of olyps orbit of Earth.
or earth rotation was measured using oppposing force
to break spin momemtun of Earth's rotation and possbly forgetting 
about the angle of rotation explaining 10 degrees offset.

Speed of sound 700m/s = N720 * 48 * 700 = false as ST =/ F
Speed of sound 720m/s = N720 * 48 * 720 = true as T = S = ST 
Speed of sound in D-algebra = (720/2)^1 = (48/2)^2 = (720/2)^3 = true

technical limitation of 48^-0: 
(48^0-1) + (48^-1 + 48^-2) + (48^-1 + 48^-2 + 48^-3) = technical limitation
therefor: ((48^0)(48^-1))((48^-1)(48^-2))((48^-1)(48^-2)(48^-3))
Inconclusion: accept -0 as value

Example to compare truancy:
pluto takes a supposed 8 months to return from it's assumed trip around 
the invisible, ftl spin, black hole at the center if the universe.
8 / 32 = 1/5 = false
8 / 48 = 1/8 = true

Inconclusion: Pluto really goes away for 8 months around (a) black hole.
Rotation of earth is a corrected (or not depending on source) 48 (per no unit).
Earth et al. get the same rotation speed(s) because they are the children (fields)
 of the black hole.relational wise.


Conclusion: Sqrt is implemented wrong for an unknown ammount of sources. 
Assume every source is wrong for almost every calculation.
Problem in use (^-2) vs. (^-1 + ^-2 + ^-3). 
Use the negative power notation sqc 
(As we live in a positive D4 this will be positive. 
Proved in my thesis the theory of dimensional fields)

Next step. Correct for Speed of light
 
Sp.o.l    = 300000 (assumed)
Sp.o.l^32 = 1,853020189x10^175
Sp.o.l.aT = 1,853020189x10???^-1 = 0
Sp.o.l.aS = 1,853020189x10???^-2 = 0
Sp.o.l.aST= 1,853020189x10???^-3 = 0

Truancy (0=1) of F^n division of 300000 km/s

Corrected
F0 = 48
Fm = S.o.L = F(48)^1 F(48)^2 F(48)^3 = 112944

Side step to affirm falsehood truancy with correcting:
S.o.l corrected for wrongly assumed spin of earth as 0.32 per half hour
Sp.o.l = (300000 / 30) * 32 = 32000


F0 = 48 is per full orbit around the sun (assumption)
F0 = (48^-1)((48^-1)(48^-2))((48^-1)(48^-2)(48^-3) = 0 = true
F0 = 48 is assumtion confirmed true


Sp.o.l = (48^-1)((48^-1)(48^-2)) = 
Sp.o.l = F1 = F((1)(48^-1))^1 F(48)^2 F(48)^3 = 2835 per full orbit 48/48
136080 per 1/48 orbit. 2 earth days is one sun (sol) day.

F(48)^1 F(48)^2 F(48)^3 = 0,000000188 
As perfect squared circle can be OR. As close a qube can be a circle. 
In context of the Sun

Assumption Sqrt and ^ cannot be used for circles or spheres.

-F0 = F
-F1 = F(48^-1)((F(48^-1)F(48^-2)) F(48^-1)(48^-2)(48^-3) = 0
-F3 = F(48^-3)((F(48^-3)F(48^-4)) F(48^-3)(48^-4)(48^-5) = 0
-ST = F(48^-1)((F(48^-1)F(48^-2)) F(1^3) = 0,000000188

0,000000188 = sol per sol day
0.000000188 / 3 = 0,000000063 sol per 1/3 sol day
0,000000063 / 3 = 0,000000021 sol per 1/6 sol day 
.. and lowest approach of s.o.l for sol

Preposition: Sol native 1 day = sol^2n1 = 136080 km/s   


(48)^-1) F(48)^-2 F(48)^-3 
F(0-1)  + F(x) F(fx) F(fx)

F0 = 48
F1 = F((1)(48^-1))^1 F(48)^2 F(48)^3 = 5308416
F2 = F(48)^1 F(2)^2 F(48)^3  = 21233664
F3 = F(48)^1 F(48)^2 F(3)^3 = 254803968

to bypass the minus 0 problem
(48^-1) + (48^-1 +( 48^-2 + 48^-3)

5308416 / 48 / 2 = 55296 per sol day
../24 = 2304
../60 = 38.4
../60 = 0,64
../2 = 0.32 = earth hours to sun hours conversion factor

S.o.l speed of light 5308,416 km/s

F0 = 32
F1 = (1)+((32^-1)(32^-2)) ((32^-1) (32^2) (32^3)

48 is sun, 24 is earth because distance around the sun takes 2 days.
48 is used as relative value. 
F0 is the default value. which can be calculated from multiple angles

Conclusion
Sqrt problmen can be found by using the Dimensional Algebra model used above.
F0 = 32, F1 = incorrect method, F2 is correct method. 

F1 = F^1 F^2 F^3 and finding the F0 value.
and fix using
F2 = F^-1 + (F^-1 + F^-2) + (F^-1 + F^-2 + F^-3)
Deduct F1 from F2 to find the offset. 
then multiply the offset to F1


The importance of the issue mostly relies on wrong input data.
32 was given for rotation but had to be adapterd to 48.
New system was used to calculate and check and was correct.

\end{verbatim}

\end{document}
..end
