\^copyright{[© R.P.N. Esseling]}
@Now this is the real light document

Light is divided into 2 basic colors, red and green going in a full circle spectrum ranging from red, to magenta/red, green, and magenta/green.
In combination with oxidation (by electron echange) the color blue appears as well from magenta. 
Poin of interest is that maganta lies directly north and south of the color spectrum and suggest possible polarity ineraction.

Light can be recorded by using a electron microscope (with blue filter) and by having nothing in it's subject recepticle besides a sideways turned red card background over a 19 hour exposure.

Radioactivity
In the nucleus radioactivity is it;s capture by radioactive light.
Outside the nucleus it's the orbiting satalites that form radioacive light.

Electron, Proton, Neutron in order when under radioactivity resemble alpha, beta and gamma radiation.
Under radioactivity is similair to the concept of static electriciy or static charged objects, in that it links all elemenets under one and exposes the nucleus to release Force.
The similairity therein is that the static elecric variant embodies larger objects than molecules alone, one could assume there is  no additional effect of underlying molecules and atoms leaking Force (or radioactivity in practice)
Suggesting a static electric blanket could fend of radiation

Acids would ground the radioacive molecule in favor of it;s nuceus, allowing alternaive discharge in non-radioctive or other-radioacive manner.

Ions could be compressed satalites of the nucleus 
