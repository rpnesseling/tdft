\^copyright{[© R.P.N. Esseling]}
@Now this is the real light document

Light is divided into 2 basic colors, red and green going in a full circle spectrum ranging from red, to magenta/red, green, and magenta/green.
In combination with oxidation (by electron echange) the color blue appears as well from magenta. 
Poin of interest is that maganta lies directly north and south of the color spectrum and suggest possible polarity ineraction.

Light can be recorded by using a electron microscope (with blue filter) and by having nothing in it's subject recepticle besides a sideways turned red card background over a 19 hour exposure.
Recorded light has shown Light has (as optional) a proton, neutron, and electron. But atleast one of those is needed.

Light is the nucleas outer without the nucleas inner and can excist without a nucleas inner. In that manner it would be a field without nucleas inner.

Radioactivity
In the nucleus radioactivity is it;s capture by radioactive light.
Outside the nucleus it's the orbiting satalites that form radioacive light. 
Radioacivity comes as light from a regenerative radioactive light source. Using nucleus Force as fuel.
(both observed as one but 1/2, outside nucleus, and inside)

Electron, Proton, Neutron in order when under radioactivity resemble alpha, beta and gamma radiation.
Under radioactivity is similair to the concept of static electriciy or static charged objects, in that it links all elemenets under one and exposes the nucleus to release Force.
The similairity therein is that the static elecric variant embodies larger objects than molecules alone, one could assume there is  no additional effect of underlying molecules and atoms leaking Force (or radioactivity in practice)
Suggesting a static electric blanket could fend of radiation

Acids will ground the nucleas outer and inner like it wouldbe a single circuit. When grounded the nucleas outer would no longer release it's light and the nucleas inner no longer disperse it's energy/force.

As light speed up past the F^3 thresholds. Light becomes Photon and F^1(+1) and Ion at F^2(+1) and ? F^3(-1). 

The diffirent patterns (wave, pulse, lineair etc.) we obseserve when meausuring light, radiofrequenty*/radiation are explained by the combinations of the elecron,proton, 
and neutron positioniong of the nucleas outer and the speed of light in F^1, F^2, and F^3. Each combination offering a diffirent pattern.

-- de/re oxigination is the same as de/re ionisation
-- fibrating radio acive force is the only one that reacts in non-oxygen (but as it has been extra-oxidized)
-- extra-oxidizaion is the drawing of oxygen into the nucleus, wich would form atomic bonds within atoms to form other, also radioactive elements.
-- speculating; radioacives' source is over-charging of the nucleus (core) and/or release of Force causing witherexchangable subatomic particles with Forced Force binding properties,
like Fields itself, chasing fields, possibly explaining the source of that behaviour on this level.
-- remember force is force with all it's possible relations on every level F\frac{F}{F} defining motion wihout cause. describing effect without cause.
-- F = F, cause and effect, definding P(aradox) for cause is time based and time could not be part of the equation. Paradox means 'yes it has a cause somewhere, somehow'
-- P == P = P\frac{P}{P}, infinite paradox or infinity. Start with neverending effect. Suggesting infinite Force without source. Likely unsustainable
-- P == P\frac{P}{P} < P\frac{P}{P}, broken paradox, therefor existence. Likely our universe because we witness heath-death and signs of Force slowing down.
-- Fn\frac{Fn}{Fn} is live or simulation thereof
-- the circel radiance decline (crd) is almost fully linked to radioactive decay. radius/radiance decline is same as degree decline.



